\documentclass[12pt]{article}
\usepackage{times,epsfig,amsmath}
\input{newcommand}
\addtolength{\oddsidemargin}{-.75in}
\addtolength{\evensidemargin}{-.75in}
\addtolength{\textwidth}{1.3in}
\addtolength{\topmargin}{-.9in}
\addtolength{\textheight}{1.5in}
%
% Change long hypen appearance
%
\def\hlinefill{\leaders\hrule height3pt depth-2.5pt\hfill}
\def\emrule{\thinspace\hbox to .75em{\hlinefill}\thinspace}
%
\makeatletter
%
% Set path for EPSFIG
%
%\define@key{Gin}{figure}{\def\Gfigname{:Figures:#1}}
%\define@key{Gin}{file}{\def\Gfigname{:Figures:#1}}
%
% Problem environment
%
\newcounter{problem}
\renewcommand{\theproblem}{F\arabic{problem}}
\newcounter{problempart}
\renewcommand{\theproblempart}{\alph{problempart}}
\newcounter{problemsubpart}
\renewcommand{\theproblemsubpart}{\roman{problemsubpart}}
\newenvironment{problems}%
{
\begin{list}%
{\bf\theproblem\hfill}%
{\usecounter{problem}\setlength{\itemindent}{-2em}\setlength{\labelwidth}{0em}}
}%
{\end{list}}
%
\newenvironment{problemparts}%
{\begin{list}%
{\bf(\theproblempart)\hfil}{\usecounter{problempart}}
}%
{\end{list}}
%
\newenvironment{problemsubparts}%
{\begin{list}%
{(\theproblemsubpart)\hfil}{\usecounter{problemsubpart}}
}%
{\end{list}}
\begin{document}
\sloppy
\begin{center}
\large\textbf{Electrical Engineering 241\\
Final\\
Due December 18, 2002}
\end{center}
\par\noindent
Three hour final exam.
Any written material created for this course can be used.
Problem~1 serves as redemption credit for Quiz~I.
Each problem is given equal weight.
Please sign the pledge when you are finished and turn in the final at my office.
\medskip\par\noindent
\rule[5pt]{2in}{.4pt}\hfill \textbf{\textit{Quiz I Redemption Problem}} \hfill \rule[5pt]{2in}{.4pt}
\begin{problems}
\item \textbf{A Simple Circuit}\\
The following simple circuit was given on a test.
\par\noindent
\centerline{\epsfig{figure=circuit34.eps}}
When the voltage source is $\sqrt{5}\sin(t)$, the current $i(t)=\sqrt{2}\cos\bigl(t-\tan^{-1}(2)-\frac{\pi}{4}\bigr)$.
\begin{problemparts}
\item
What is voltage $v_{\textrm{out}}(t)$?
\item
What is the impedance $\impedance$ at the frequency of the source?
\end{problemparts}
\hspace{-\leftmargin}\rule[5pt]{\textwidth}{.4pt}
%***************
\item \textbf{Signal Scrambling}\\
An excited inventor announces the discovery of a way of using analog technology to render music unlistenable without knowing the secret recovery method.
The idea is to modulate the bandlimited message $\msg(t)$ by a special periodic signal $\signal(t)$ that is zero during half of its period, which renders the message unlistenable and superficially, at least, unrecoverable.
\par\noindent
\centerline{\epsfig{figure=sig49.eps}}
\begin{problemparts}
\item
What is the Fourier series for the periodic signal?
\item
What are the restrictions on the period $\period$ so that the message signal can be recovered from $\msg(t)\cdot\signal(t)$?
\item
ELEC~241 students think they have ``broken'' the inventor's scheme and are going to announce it to the world.
How would they recover the original message \emph{without} having detailed knowledge of the modulating signal?
\end{problemparts}
\clearpage
%***************
\item \textbf{Complementary Filters}\\
\emph{Complementary filters} have transfer functions that add to one.
Mathematically, $\transfer_1(f)$ and $\transfer_2(f)$ are complementary if
\[
\transfer_1(f)+\transfer_2(f)=1\;.
\]
We can use complementary filters to separate a signal into two parts by passing it through each filter.
Each output can then be transmitted separately and the original signal reconstructed at the receiver.
Let's assume the message is bandlimited to $W$~Hz and that $\transfer_1(f)=\frac{a}{a+j2\pi f}$.
\begin{problemparts}
\item
What circuits would be used to produce the complementary filters?
\item
Sketch a block diagram for a communication system (transmitter and receiver) that employs complementary signal transmission to send a message $\msg(t)$.
\item
What is the receiver's signal-to-noise ratio?
How does it compare to the standard system that sends the signal by simple amplitude modulation?
\end{problemparts}
%*********
\item \textbf{Simulating the Real World}\\
Much of physics is governed by differential equations, and we want to use signal processing methods to simulate physical problems.
The idea is to replace the derivative with a discrete-time approximation and solve the resulting difference equation.
For example, suppose we have the differential equation
\[
\dfrac{d\outsig(t)}{dt}+a\outsig(t)=\insig(t)
\]
and we approximate the derivative by
\[
\left.\dfrac{d\outsig(t)}{dt}\right|_{t=n\period} \approx \dfrac{\outsig(n\period)-\outsig\bigl((n-1)\period\bigr)}{\period}
\]
where $\period$ essentially amounts to a samping interval.
\begin{problemparts}
\item
What is the difference equation that must be solved to approximate the differential equation?
\item
When $\insig(t)=\step(t)$, the unit step, what will be the simulated output?
\item
Assuming $\insig(t)$ is a sinusoid, how should the sampling interval $\period$ be chosen so that the approximation works well?
\end{problemparts}
\clearpage
%*********
\item \textbf{Repeaters}\\
Because signals attenuate with distance from the transmitter, \emph{repeaters} are frequently employed for both analog and digital communication.
For example, let's assume that the transmitter and receiver are $D$~m apart, and a repeater is positioned halfway between them.
What the repeater does is amplify its received signal to exactly cancel the attenuation encountered along the first leg and to re-transmit the signal to the ultimate receiver.
However, the signal the repeater receives contains white noise as well as the transmitted signal.
The receiver experiences the same amount of white noise as the repeater.
\par\noindent
\centerline{\epsfig{figure=sys35.eps}}
\begin{problemparts}
\item
What is the block diagram for this system?
\item
For an amplitude-modulation communication system, what is the signal-to-noise ratio of the demodulated signal at the receiver?
Is this better or worse than the signal-to-noise ratio when no repeater is present?
\item
For digital communication, we must consider the system's capacity.
Is the capacity larger with the repeater system than without it?
If so, when;
if not, why not?
\end{problemparts}
%***************
\end{problems}
\end{document}